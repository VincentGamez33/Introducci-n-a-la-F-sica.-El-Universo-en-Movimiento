\chapternn{PREFACIO}
\chapter*{PREFACIO}
\addcontentsline{toc}{chapter*}{Prefacio}
\chaptermark{Prefacio}

Este texto es el fruto de un viaje apasionante por el mundo de la física, que nos transporta desde los fundamentos de la mecánica clásica hasta los intrincados fenómenos del electromagnetismo. Inspirado en las detalladas y minuciosas notas del Profesor Adolfo Helmut Navarro Rudolf, distinguido académico de la ESFM, este trabajo se propone transmitir de forma clara y cercana los conceptos que han marcado el avance de la disciplina. Sus aportaciones no solo han contribuido al enriquecimiento académico de la institución, sino que también han encendido la llama del asombro y la curiosidad en cada estudiante que se adentra en el estudio de esta fascinante ciencia.

La elaboración de esta obra ha sido un proceso tan enriquecedor como retador. Bajo mi autoría, he procurado organizar y presentar el contenido de manera que resulte accesible y estimulante para estudiantes y aficionados por igual. Mi intención es motivar una mirada crítica y apasionada hacia los misterios del universo, haciendo eco de la excelencia que caracteriza a la ESFM y a sus distinguidos miembros.

No puedo dejar de expresar mi más sincero agradecimiento a \href{https://github.com/antobno}{\color{mainc}Marco Antonio Molina Mendoza}, también perteneciente a la ESFM, cuyos comandos y entornos de \LaTeX\ ayudaron para mejorar el presente trabajo. Gracias a su soporte técnico, la presentación del material refleja la dedicación, el rigor y la pasión que la materia merece, consolidando así el compromiso de nuestra comunidad académica con la excelencia.

Con la esperanza de que esta obra inspire, motive y facilite el camino en el aprendizaje de la física, le invito a disfrutar y descubrir los innumerables encantos y desafíos que esta fascinante disciplina nos ofrece. Cada página es un reflejo del empeño y la pasión compartida por quienes, en la ESFM, día a día trabajan para expandir los horizontes del conocimiento científico.% \marginElement{
%     \includegraphics[width=\marginparwidth]{images/prefacio/VG.pdf}
% }

\vspace{2ex}

\begin{flushright}
    \cafe\selectfont\itshape Vicente C. Gámez \\
    México, 2025.
\end{flushright}\marginElement{
\begin{center}
    \begin{tikzpicture}
        \node[fill=white] at (0,0) {\hypersetup{hidelinks}\qrcode[hyperlink, height=0.8\linewidth]{https://github.com/VincentGamez33}};
    \end{tikzpicture}
\end{center}
}